\documentclass[a4paper,12pt]{report}
\usepackage[slovene]{babel}
\usepackage[utf8]{inputenc}
\usepackage[T1]{fontenc}
\usepackage{lmodern}

\usepackage{url}
\usepackage{graphicx}
\usepackage{amsmath}
\usepackage{amsthm}
\usepackage{booktabs}
\usepackage{xcolor}
\usepackage{physics}
\usepackage{amsfonts}
\usepackage{mathrsfs}

{\theoremstyle{plain}
\newtheorem{izrek}{Izrek}[section]
\newtheorem{posledica}[izrek]{Posledica}
}

{\theoremstyle{plain}
\newtheorem{lema}{Lema}
}

{\theoremstyle{definition}
\newtheorem{definicija}[izrek]{Definicija}
\newtheorem{vaja}[izrek]{Vaja}
}

{\theoremstyle{plain}
\newtheorem{trditev}{Trditev}
}

\begin{document}

\title{Priprave na Analiza 1 ustni izpit}
\author{Mia Žele}
\date{}

\maketitle

\tableofcontents

\begin{abstract}
    Dokument je namenjen pripravam na ustni izpit. Tu bodo zbrani najpomembnejši
    izreki, trditve in definicije, ki jih bo treba znati dokazati.
\end{abstract}


\chapter{Števila}

\section*{Pomembne definicije in izreki}

\pagebreak

\section*{Naloge}


\chapter{Zaporedja}

\section*{Pomembne definicije in izreki}

\pagebreak

\section*{Naloge}

\begin{enumerate}
    \item Definicija zaporedja realnih števil.
    \item Omejenost zaporedja.
    \item Definicija konvergence, limite zaporedja.
    \item Konvergentno zaporedje ima eno samo limito.
    \item Konvergentno zaporedje je omejeno.
    \item Stekališče zaporedja.
    \item Če vsaka okolica števila $s \in \mathbb{R}$ vsebuje člen zaporedja, ki je različen od $s$, potem je $s$ stekališče.
    \item Vsako omejeno zaporedje ima stekališče. \\ (Ideja: $\mathcal{U} = \{ u \in \mathbb{R}; a_n < u \text{ za kvečjemu končno mnogo indeksov n} \}$)
    \item Kdaj je zaporedje monotono?
    \item Monotono zaporedje je Konvergentno natanko tedaj, kadar je omejeno.
    \item Definicija podzaporedja, rep zaporedja.
    \item Će zaporedje konvergira k $a$, potem tudi vsako njegovo podzaporedje konvergira k $a$.
    \item Število $s \in \mathbb{R}$ je stekališče zaporedja natanko tedaj, kadar obstaja podzaporedje, ki konvergira proti $s$.
    \item Računanje z zaporedji.
    \item Izrek o sendviču.
    \item Cauchyjev pogoj.
    \item Zaporedje je konvergentno natanko tedaj, ko je Cauchyjevo.
    \item Omejeno zaporedje z enim samim stekališčem je konvergentno.
    \item Definicija limite v $\infty$.
    \item Definicija zgornje in spodnje limite. (plus dve trditvi)
    \item Če $\abs{a} < 1$, potem $(a^n)_n$ konvergira proti 0.
    \item $x > 0: \text{ } \lim_{n \to \infty} \sqrt[n]{x} = 1$
    \item $\lim_{n \to \infty} \sqrt[n]{n} = 1$
    \item $e = \lim_{n \to \infty}(1 + \frac{1}{n})^n$
    \item Definicija potence pri realnem eksponentu.
    \item $a > 0: \lim_{n \to \infty} \frac{1}{n^a} = 0$
    \item $a,q \in \mathbb{R}, q > 1: \lim_{n \to \infty} \frac{n^a}{q^n} = 0$
    \item Zaporedja kompleksnih števil. Kdaj konvergira? Cauchyjev pogoj. Pravila za računanje.
\end{enumerate}


\chapter{Vrste}

\section*{Pomembne definicije in izreki}

\pagebreak

\section*{Naloge}

\begin{enumerate}
    \item definicija konvergence vrste
    \item konvergenca geometrijske vrste
    \item trditev povezana s konvergenco vrste
    \item harmonična vrsta
    \item primerjalni kriterij
    \item $ \sum_{n=1}^{\infty} \frac{1}{n^p} $
    \item kvocietni kriterij
    \item korenski ali Cauchyjev kriterij
    \item Raabejev kriterij
    \item definicija absolutna in pogojna konvergenca
    \item če vrsta abs. konvergira, potem je konvergentna
    \item Leibnizov kriterij za alternirajoče vrste
    \item preureditve vrste
    \item če je vrsta absolutno konvergentna, potem je preureditev tudi konvergentna
    \item produkt vrst
    \item dvakratne vrste
\end{enumerate}

\chapter{Funkcije in zveznost}

\section*{Funkcije realne spremenljivke}

\subsection*{Pomembne definicije in izreki}



\subsection*{Naloge}

\begin{enumerate}
    \item Kdaj sta funkciji enaki?
    \item Zožitev funkcije.
    \item Katere podmnožice $S \subset \mathbb{R} \times \mathbb{R}$ so grafi?
    \item Komponiranje funkcij
    \item Kaj je inverzna funkcija od funkcije $f$?
    \item Graf inverzne funkcije.
    \item Kdaj je funkcija (navzgor/navzdol) omejena?
    \item Supremum, infimum, maksimum, minimum.
    \item Osnovne operacije s funkcijami.
\end{enumerate}

\pagebreak

\section*{Zveznost}


\subsection*{Pomembne definicije in izreki}
Naslednje tri definicije so si ekvivalentne.
\begin{definicija}
    Naj bo $D \subset \mathbb{R}$, $f: D \to \mathbb{R}$ funkcija in $a \in D$. 
    Funkcija $f$ je zvezna v točki $a$, če za vsak $\varepsilon>0$ obstaja $\delta>0$, da 
    za vse $x \in D$ velja: če je $\abs{x-a} < \delta$, potem $\abs{f(x)-f(a)} < \varepsilon$
\end{definicija}

\begin{definicija}
    $f$ je zvezna v točki $a \in D$, če za vsak $\varepsilon > 0$ obstaja $\delta > 0$, 
    da se $\delta$-okolica točke $a$ s funkcijo $f$ preslika v $\varepsilon$-okolico točke $f(a)$.
\end{definicija}

\begin{definicija}
    $f$ je zvezna v točki $a \in D$, če za vsako okolico $\mathcal{V}$ od $f(a)$ obstaja okolica 
    $\mathcal{U}$ od $a$ v $D$, da velja: $f(\mathcal{U}) \subset \mathcal{V}$
\end{definicija}


\begin{izrek}
    Naj bo $D \subset \mathbb{R}, f: D \to \mathbb{R}$ funkcija, $a \in D$. Potem je funkcija $f$ 
    zvezna v točki $a$ natanko tedaj, kadar za vsako zaporedje $(x_n)$ v $D$, ki konvergira proti $a$, 
    zaporedje $(f(x_n))$ konvergira proti $f(a)$.
\end{izrek}

\subsection*{Naloge}

\begin{enumerate}
    \item Karakterizacija zveznosti, grafični prikaz, primeri nezvenzih funkcij.
    \item Primerjava različnih definicij zveznosti: standardna, z okolicami.
    \item Dirichletova funkcija.
    \item Opis zveznosti z zaporedji.
    \item Če sta $f$ in $g$ zvezni v $a$, potem so tudi $f+g, f-g, fg $ in $\frac{f}{g} (g(x) \neq 0)$ zvezne v $a$.
    \item Zveznost kompozituma.
    \item Kdaj pravimo, da je funkcija zvezna?
    \item Definicija limite funkcije.
    \item Leva in desna limita.
    \item Definicija (strogo) naraščajoče/padajoče funkcije; kaj je skok funkcije?
    \item Monotona funkcija definirana na zaprtem intervalu ima kvečjemu končno mnogo točk nezvevnosti.
    \item Limita funkcije, ko gre $x$ preko vsake meje.
    \item Cauchyjev pogoj.
    \item $\lim_{x \to 0} \frac{\sin(x)}{x}$
    \item $\lim_{x \to 0} \frac{a^x - 1}{x}$
    \item Definicija enakomerne zveznosti.
    \item Zvezna funkcija definirana na zaprtem intervalu je enakomerno zvezna.
    \item Lema o pokritjih. (s tem dokažeš zgornji izrek)
    \item Bisekcija. Izrek o vloženih intervalih.
    \item Zvezna funkcija definirana na zaprtem intervalu je omejena (doseže minimum in maksimum).
    \item Inverz od strogo monotone zvezne funkcije definirane na zaprtem intervalu je zvezen.
    \item Zveznosti posebnih funkcij: eksponentna, logaritemska, kotne, ciklometrične.
\end{enumerate}





\chapter{Odvod}

\section*{Pomembne definicije in izreki}

Vse izreke in trditve je treba znati dokazati!

\begin{definicija}
    Naj bo funkcija $f$ definirana v okolici točke $a$. Če obstaja limita
     $\lim_{h\to 0}\frac{f(a+h)-f(a)}{h}$, jo imenujemo odvod funkcije $f$ v 
     točki $a$ in jo označimo s $f'(a)$ in rečemo, da je $f$ odvedljiva v točki $a$.
\end{definicija}

\begin{izrek}
    Naj bo funkcija $f$ definirana v okolici točke $a$. Če je $f$ odvedljiva v $a$, 
    potem je $f$ zvezna v točki $a$.
\end{izrek}

\begin{trditev}
    Naj bo funkcija $f$ definirana v okolici točke $a$. Funkcija $f$ je odvedljiva v točki $a$
     natanko tedaj, kadar obstajata levi in desni odvod funkcije $f$ v točki $a$ in sta enaka.
\end{trditev}

\begin{izrek}
    Naj bo funkcija $f$ definirana v okolici točke $a$. Potem je $f$ diferenciabilna v točki $a$ 
    natanko tedaj, kadar je $f$ odvedljiva v točki $a$. Tedaj velja: $$\dd f(a)(h)=f'(a) h$$ 
\end{izrek}

\begin{izrek}
    Naj bo funkcija $f: [a,b] \to \mathbb{R}$ odvedljiva v točki $c \in (a,b)$. 
    Če je $c$ lokalni ekstrem funkcije $f$, potem $f'(c)=0$. 
\end{izrek}

\begin{izrek}
    \emph{Rollov izrek}: Naj bo funkcija $f: [a,b] \to \mathbb{R}$ zvezna na $[a,b]$ in odvedljiva na $(a,b)$. 
    Če je $f(a)=f(b)$, potem obstaja $c \in (a,b): f'(c)=0$.
\end{izrek}

\begin{izrek}
    \emph{Lagrangeev izrek}: Naj bo funkcija $f: [a,b] \to \mathbb{R}$ zvezna na $[a,b]$ in odvedljiva na $(a,b)$. 
    Potem obstaja $c \in (a,b)$, da je $f(b)-f(a)=f'(c)(b-a)$.
\end{izrek}

\begin{trditev}
    Denimo,da je funkcija $f$ dvakrat odvedljiva v okolici točke $a$ in denimo, da je $a$ stacionarna točkka od $f$.
    \begin{enumerate}
        \item Če je $f''(x) \leq 0$ za vse $x$ v neki okolici točke $a$, potem ima $f$ v $a$ lokalni maksimum.
        \item Če je $f''(x) \geq 0$ za vse $x$ v neki okolici točke $a$, potem ima $f$ v $a$ lokalni minimum.
    \end{enumerate}
\end{trditev}

\begin{definicija}
    Naj bo funkcija $f$ definirana na intervalu $I$. Pravimo, da je $f$ \textbf{konveksna} na $I$, če velja: za vsak $a,b \in I, a < b$ velja: 
    $$f(x) \leq f(a) + \frac{f(b)-f(a)}{b-a} (x-a) \text{ za vse } x \in [a,b].$$
\end{definicija}

\begin{definicija}
    Naj bo funkcija $f$ definirana na intervalu $I$. Pravimo, da je $f$ \textbf{konkavna} na $I$, če velja: za vsak $a,b \in I, a < b$ velja: 
    $$f(x) \geq f(a) + \frac{f(b)-f(a)}{b-a} (x-a) \text{ za vse } x \in [a,b].$$
\end{definicija}

\begin{izrek}
    Naj bo $f$ odvedljia funkcija na odprtem intervalu $I$. Potem je $f$ konveksna na $I$ natanko tedaj, 
    kadar za vsaka $a,x \in I$ velja: $$f(x) \geq f(a) + f'(a)(x-a).$$
\end{izrek}

\begin{izrek}
    Naj bo $f$ odvedljiva na odprtem intervalu $I$. Potem je $f$ konveksna natanko tedaj, 
    kadar je odvod $f'$ naraščajoča na $I$. Če je $f$ dvakrat odvedljiva na $I$, potem je $f$ konveksna natanko tedaj, 
    kadar je $f'' \geq 0$ na $I$.
\end{izrek}

\begin{definicija}
    Naj bo funkcija $f$ definirana na intervalu $I$. Če za $a \in I$ obstaja okolica točke $a$, da je na eni strani $f$ 
    konveksna, na drugi točke $a$ znotraj te okolice pa $f$ konkavna, potem rečemo, da ima $f$ v $a$ \textbf{prevoj}. 
\end{definicija}

\begin{lema}
    Cauchyjev izrek: Naj bosta $f$ in $g$ zvezni funkciji na $[a,b]$, odvedljivi na $(a,b)$ in naj velja, da  
    $g'(x) \neq 0 \text{ za vse } x \in (a,b)$. Tedaj obstaja $c \in  (a,b)$: $$\frac{f(b)-f(a)}{g(b)-g(a)} = \frac{f'(c)}{g'(c)}$$
\end{lema}

\begin{izrek}
    \textbf{L'Hôpitalovo pravilo}: Naj bosta funkciji $f$ in $g$ odvedljivi na $(a,b)$ in denimo, da velja:
    \begin{enumerate}
        \item $g(x) \neq 0, g'(x) \neq 0 \text{ za vse } x \in (a,b)$,
        \item $\lim_{x \searrow a} f(x) = 0 \text{ in } \lim_{x \searrow a} g(x) = 0$.
    \end{enumerate} 
    Če obstaja $\lim_{x \searrow a} \frac{f'(x)}{g'(x)}$, potem obstaja $\lim_{x \searrow a} \frac{f(x)}{g(x)}$ in limiti sta enaki.
\end{izrek}

\begin{izrek}
    Naj bosta funkciji $f$ in $g$ odvedljivi na $(a,b)$, naj bo $g'(x) \neq 0 \text{ za vse } x \in (a,b) \text{ in } \lim_{x \searrow a} g(x) = \infty$ (ali $- \infty$). 
    Če obstaja $\lim_{x \searrow a} \frac{f'(x)}{g'(x)}$, potem obstaja $\lim_{x \searrow a} \frac{f(x)}{g(x)}$ in sta enaki.
\end{izrek}

\begin{definicija}
    Pot v ravnini je preslikava $F: I \to \mathbb{R}^2$, kjer je $I$ interval, $x \in I: F(x) = (\alpha(x), \beta(x))$, in kjer sta funkciji $\alpha$ in $\beta$ 
    zvezni. Tir poti je množica $K = F(I) = {(\alpha(x), \beta(x)); x \in I}$. Preslikavo $F$ imenujemo parametrizacija tira poti $K$.
\end{definicija}

\begin{izrek}
    Naj bo $F: I \to \mathbb{R}^2$ zvezno odvedljiva pot, $t_0$ iz notranjosti intervala $I$ in denimo, da je $\dot{F}(t_0) \neq 0$. Če je $\dot{\alpha}(t_0) \neq 0$, 
    potem obstaja tak $\delta > 0$, da lahko tir poti $K = {F(t); \abs{t - t_0} < \delta}$ zapišemo kot graf neke odvedljive funkcije $f$ nad intervalom $U$ 
    okrog točke $\alpha(t_0)$: $K = {(x, f(x)); x \in U}$. Velja: $f'(\alpha(t)) = \frac{\dot{\beta(t)}}{\dot{\alpha(t)}}, \abs{t - t_0} < \delta$.
\end{izrek}


\pagebreak

\section*{Naloge}
\begin{enumerate}
    \item Karakterizacija odvoda.
    \item Splošna definicija. Kaj je diferenčni kvocient? Geometrijski pomen.
    \item Potreben pogoj za odvedljivost.
    \item Levi in desni odvod.
    \item Kdaj pravimo, da je funkcija odvedljiva na zaprtem/odprtem intervalu?
    \item Kdaj je funkcija zvezno odvedljiva?
    \item Kdaj je funkcija odsekoma zvezno odvedljiva?
    \item Aproksimacija $ f(a+h) - f(a) \approx f'(a) h $
    \item Kdaj je funkcija $f$ diferenciabilna v točki $a$? Kaj je diferencial funkcije?
    \item Pravila za odvajanje: odvod vsote, produkta, kvocienta, kompozituma funkcij.
    \item Odvod inverza.
    \item Odvodi elementarnih funkcij.
    \item Odvodi višjega reda.
    \item Rollov in Lagrangeev izrek, povezava med njima in njegove posledice.
    \item Lokalni ekstremi.
    \item Stacionarna točka.
    \item Iskanje globalnih ekstremov odvedljivih funkcij, kandidati.
    \item Definicija konveksnosti in konkavnosti.
    \item Kako rišemo grafe funkij s pomočjo odvoda?
    \item L'Hôpitalovo pravilo in njegova uporaba.
    \item Kako lahko podajamo krivulje?
    \item Kaj je pot v ravnini, tir poti, parametrizacija?
    \item Kdaj pravimo, da je pot zvezno odvedljiva?
\end{enumerate}




\chapter{Integral}

\section*{Pomembne definicije in izreki}

\pagebreak

\section*{Naloge}


\chapter{Funkcijska zaporedja in vrste}

\section*{Pomembne definicije in izreki}

\pagebreak

\section*{Naloge}


\chapter{Metrični prostori}

\section*{Pomembne definicije in izreki}

\pagebreak

\section*{Naloge}



\end{document}